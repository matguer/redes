\par Para este segundo experimento tomamos nuestras mediciones en una red laboral.

\subsubsection{Fuente $S$}

\par Veamos entonces las m\'etricas de la fuente $S$ propuesta modelada con los resultados del experimento: \\

\begin{tabular}{ | c | c | c |}
    \hline
    Mensaje & Probabilidad & Información [bits] \\
    \hline
    \textit{Unicast} & 0.464 & 1.109 \\
    \hline
    \textit{Broadcast} & 0.536 & 0.898 \\
    \hline
\end{tabular} \\

\par Entropía de la fuente: 0.996 bits. Entropía máxima: 1 bit.

\par En este caso podemos observar como las probabilidades de aparici\'on de los dos tipos de paquetes rondan el 50\%, es decir, sus apariciones son pr\'acticamente equiprobables. \\
De esto podemos deducir y comprobar matem\'aticamente que los simbolos de la fuente $S$ casi no nos brindan informaci\'on. Por tal m\'otivo, si observamos la entrop\'ia de la misma, vamos a notar que se acerca demasiado a 1 bit, es decir, la entrop\'ia m\'axima, lo cual tiene sentido ya que sabemos que este valor se alcanza cuando los eventos son equiprobables, y en nuestro caso, las apariciones de los simbolos de $S$ se asemejan mucho a ese escenario.