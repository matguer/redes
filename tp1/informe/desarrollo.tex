\section{Desarrollo}

\subsection{Fuente $S$}

\par Definiremos una fuente de memoria nula $S$ en base a los frames de capa de enlace capturados. 
La fuente consiste en dos mensajes: un frame fue transmitido de forma \textit{broadcast}, o éste fue transmitido de forma \textit{unicast}.

\subsection{Elección de la fuente $S_1$}

\par Definiremos una fuente de memoria nula $S_1$ en base a las Direcciones IP de los paquetes ARP. 
Deberemos tomar diversas decisiones para definirla correctamente para poder distinguir los nodos apropiados.

\par En primer lugar: debemos elegir si tomar los paquetes \textit{who-has}, \textit{is-at}, o ambos.
En la mayoría de los casos, un \textit{who-has} será respondido por exactamente un \textit{is-at} correspondiente, a menos que el receptor deseado no pueda recibir el paquete o emitir la respuesta, o que haya dos dispositivos con una misma MAC address que intenten responder a la vez.
Por ende, la información del \textit{is-at} será redundante con la del \textit{who-has}, a menos que se produzca un error (lo que, de tomar ambos, agregaría errores a las mediciones).

\par Consecuentemente, tomaremos sólo uno.
Ya que el \textit{who-has} se transmite de forma \textit{broadcast}, mientras que el \textit{is-at}, de forma \textit{unicast}\footnote{Si bien realizaremos las mediciones en modo promiscuo, la presencia de \textit{switches} puede evitar que veamos este tipo de paquetes si no están destinados a nuestro dispositivo, lo que generaría aún más errores en las mediciones.}, tomaremos el primero.

\par En segundo lugar, debemos decidir si emplear el origen del \textit{who-has}, su destino, o ambos como el mensaje de la fuente.
Esta decisión no la tomaremos de antemano, sino que observaremos los grafos resultantes de los experimentos y en base a ellos decidiremos cuál es la opción más acertada.

\par En último lugar, debemos decidir si permitir mensajes repetidos\footnote{Es decir, si considerar repetidas veces múltiples paquetes ARP con igual origen y destino.}. 
Si bien esto no es ilógico desde el punto de vista del modelo de fuente de memoria nula planteado, los paquetes ARP repetidos no deberían ser necesarios: una vez que se envía un \textit{who-has} por una cierta dirección IP, y éste es respondido por un \textit{is-at}, la relación entre esta dirección y la MAC address provista debería persistirse en una tabla del emisor; paquetes repetidos podrían ser síntomas de que el \textit{who-has} original no tuvo respuesta, por lo que otros posteriores fueron requeridos.

\par Creemos que por esta razón no deberíamos considerar paquetes repetidos, pero de todas formas juzgaremos ambos procedimientos en base a los resultados de los experimentos.

\par Los grafos que emplearemos para representar la red subyacente de mensajes ARP serán independientes de las elecciones que tomemos respecto de la fuente.
En particular, éstos consistirán en digrafos con loops\footnote{Como mencionamos en la Introducción, en un \textit{gratuitous ARP}, $\text{\textit{Sender's Protocol Address}} = \text{\textit{Target Protocol Address}}$. Para poder observar este fenómeno en el grafo, permitiremos loops.}, donde hay un eje de un nodo a otro si el primero emite un \textit{who-has} preguntando por la Dirección IP del segundo. 

\subsection{Experimento 1: red inalámbrica de los laboratorios del DC}

\par Para este experimento evaluamos la red inalámbrica de los laboratorios del DC.

\subsubsection{Estructura de la red}

\par En las figuras \ref{ARPDC-sinColapsar} y \ref{ARPDC} se pueden ver los grafos\footnote{Ambos grafos representan la misma red. Sin embargo, el tamaño del grafo \ref{ARPDC-sinColapsar} puede dificultar un análisis detallado, por lo que en la figura \ref{ARPDC} colapsamos ciertos nodos con iguales vecinos (que se muestran en rojo). Mantuvimos el grafo original ya que una mirada rápida ofrece más información concerniente a la topología de la red.} de la red subyacente de mensajes ARP.

\begin{figure*}[ht]
    \centering
    \includegraphics[width=0.9\textwidth]{figuras/ciudad_10_grafoSinColapsar.pdf}
    \caption{Grafo de la red subyacente de mensajes ARP, sin colapsar nodos .}\label{ARPDC-sinColapsar}
\end{figure*}

\begin{figure*}[ht]
    \centering
    \includegraphics[width=0.9\textwidth]{figuras/ciudad_10_grafo.pdf}
    \caption{Grafo de la red subyacente de mensajes ARP, colapsando nodos.}\label{ARPDC}
\end{figure*}

\par Si bien estas figuras fueron creadas a partir de los paquetes ARP, que se relacionan más con la fuente $S_1$, nos dan una idea de la estructura de la red, lo que nos permitirá mejor comprender los resultados de la fuente $S$.

\subsubsection{Fuente $S$}

\par A continuación podemos ver la fuente $S$ propuesta modelada con los resultados del experimento: \\

\begin{tabular}{ | c | c | c |}
    \hline
    Mensaje & Probabilidad & Información [bits] \\
    \hline
    \textit{Unicast} & 0.773 & 0.371 \\
    \hline
    \textit{Broadcast} & 0.227 & 2.141 \\
    \hline
\end{tabular} \\

\par Entropía de la fuente: 0.772 bits. Entropía máxima: 1 bit.

\par Observamos que la entropía de la fuente es menor que la máxima, ya que las transmisiones \textit{unicast} son casi 3 veces más probables que las \textit{broadcast}.
Si bien no poseemos una idea previa de cómo debería comportarse esta fuente, vemos que la entropía y la probabilidad de los frames \textit{broadcast} son significativamente mayores para la fuente $S$ en el experimento 2.

\par Podemos ver, comparando la estructura de ambas redes (figuras \ref{ARPDC-sinColapsar} y \ref{ARPTrab-sinColapsar}, respectivamente), que la de este experimento es más fragmentaria, mientras que la del 2 posee un mucho mayor grado de interconexión.

\subsubsection{Fuente $S_1$}

Incluir por lo menos los siguientes gr\'aficos:

\begin{itemize}
	\item Dada la fuente binaria S, mostrar la cantidad de infomación de cada símbolo comparando con la entropía de la fuente y la entropía máxima
	\item Dados los paquetes ARP, muestre mediante un grafo, la red de mensajes ARP subyacente
	\item Dada la fuente S1, mostrar la cantidad de información de cada símbolo comparando con la entropía de la fuente.
\end{itemize}

Responder las siguientes preguntas (solo dejar las respuestas):

\begin{itemize}
	\item ¿La entropía de la fuente S es máxima? ¿Que sugiere esto acerca de la red? ¿Está relacionado con el overhead impuesto por la red debido a los protocolos de control (i.e.: ARP)?
	\item ¿Cómo es el tráVco ARP en la red? ¿Se pueden distinguir nodos? ¿Cuántos? ¿Indica algo la cantidad? ¿Se les puede adjudicar alguna función especíVca? ¿Hay evidencia parcial que sugiera que algún nodo funciona de forma anómala y/o no esperada?
	\item ¿Existe una correspondencia entre lo que se conoce de la red y los nodos distinguidos detectados por la herramienta? ¿Es posible usar el criterio de distinción propuesto como método para descubrir el/los Default Gateway/s de la red? ¿Es preciso?
\end{itemize}

\subsection{Experimento 2}

Descripci\'on del experimiento y de las condiciones del experimento

Incluir por lo menos los siguientes gr\'aficos:

\begin{itemize}
	\item Dada la fuente binaria S, mostrar la cantidad de infomación de cada símbolo comparando con la entropía de la fuente y la entropía máxima
	\item Dados los paquetes ARP, muestre mediante un grafo, la red de mensajes ARP subyacente
	\item Dada la fuente S1, mostrar la cantidad de información de cada símbolo comparando con la entropía de la fuente.
\end{itemize}

Responder las siguientes preguntas (solo dejar las respuestas):

\begin{itemize}
	\item ¿La entropía de la fuente S es máxima? ¿Que sugiere esto acerca de la red? ¿Está relacionado con el overhead impuesto por la red debido a los protocolos de control (i.e.: ARP)?
	\item ¿Cómo es el tráVco ARP en la red? ¿Se pueden distinguir nodos? ¿Cuántos? ¿Indica algo la cantidad? ¿Se les puede adjudicar alguna función especíVca? ¿Hay evidencia parcial que sugiera que algún nodo funciona de forma anómala y/o no esperada?
	\item ¿Existe una correspondencia entre lo que se conoce de la red y los nodos distinguidos detectados por la herramienta? ¿Es posible usar el criterio de distinción propuesto como método para descubrir el/los Default Gateway/s de la red? ¿Es preciso?
\end{itemize}


\subsection{Experimento 3}

Descripci\'on del experimiento y de las condiciones del experimento

Incluir por lo menos los siguientes gr\'aficos:

\begin{itemize}
	\item Dada la fuente binaria S, mostrar la cantidad de infomación de cada símbolo comparando con la entropía de la fuente y la entropía máxima
	\item Dados los paquetes ARP, muestre mediante un grafo, la red de mensajes ARP subyacente
	\item Dada la fuente S1, mostrar la cantidad de información de cada símbolo comparando con la entropía de la fuente.
\end{itemize}

Responder las siguientes preguntas (solo dejar las respuestas):

\begin{itemize}
	\item ¿La entropía de la fuente S es máxima? ¿Que sugiere esto acerca de la red? ¿Está relacionado con el overhead impuesto por la red debido a los protocolos de control (i.e.: ARP)?
	\item ¿Cómo es el tráVco ARP en la red? ¿Se pueden distinguir nodos? ¿Cuántos? ¿Indica algo la cantidad? ¿Se les puede adjudicar alguna función especíVca? ¿Hay evidencia parcial que sugiera que algún nodo funciona de forma anómala y/o no esperada?
	\item ¿Existe una correspondencia entre lo que se conoce de la red y los nodos distinguidos detectados por la herramienta? ¿Es posible usar el criterio de distinción propuesto como método para descubrir el/los Default Gateway/s de la red? ¿Es preciso?
\end{itemize}
