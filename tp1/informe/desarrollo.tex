\section{Desarrollo}

\subsection{Experimento 1}

Descripci\'on del experimiento y de las condiciones del experimento

Incluir por lo menos los siguientes gr\'aficos:

\begin{itemize}
	\item Dada la fuente binaria S, mostrar la cantidad de infomación de cada símbolo comparando con la entropía de la fuente y la entropía máxima
	\item Dados los paquetes ARP, muestre mediante un grafo, la red de mensajes ARP subyacente
	\item Dada la fuente S1, mostrar la cantidad de información de cada símbolo comparando con la entropía de la fuente.
\end{itemize}

Responder las siguientes preguntas (solo dejar las respuestas):

\begin{itemize}
	\item ¿La entropía de la fuente S es máxima? ¿Que sugiere esto acerca de la red? ¿Está relacionado con el overhead impuesto por la red debido a los protocolos de control (i.e.: ARP)?
	\item ¿Cómo es el tráVco ARP en la red? ¿Se pueden distinguir nodos? ¿Cuántos? ¿Indica algo la cantidad? ¿Se les puede adjudicar alguna función especíVca? ¿Hay evidencia parcial que sugiera que algún nodo funciona de forma anómala y/o no esperada?
	\item ¿Existe una correspondencia entre lo que se conoce de la red y los nodos distinguidos detectados por la herramienta? ¿Es posible usar el criterio de distinción propuesto como método para descubrir el/los Default Gateway/s de la red? ¿Es preciso?
\end{itemize}

\subsection{Experimento 2}

Descripci\'on del experimiento y de las condiciones del experimento

Incluir por lo menos los siguientes gr\'aficos:

\begin{itemize}
	\item Dada la fuente binaria S, mostrar la cantidad de infomación de cada símbolo comparando con la entropía de la fuente y la entropía máxima
	\item Dados los paquetes ARP, muestre mediante un grafo, la red de mensajes ARP subyacente
	\item Dada la fuente S1, mostrar la cantidad de información de cada símbolo comparando con la entropía de la fuente.
\end{itemize}

Responder las siguientes preguntas (solo dejar las respuestas):

\begin{itemize}
	\item ¿La entropía de la fuente S es máxima? ¿Que sugiere esto acerca de la red? ¿Está relacionado con el overhead impuesto por la red debido a los protocolos de control (i.e.: ARP)?
	\item ¿Cómo es el tráVco ARP en la red? ¿Se pueden distinguir nodos? ¿Cuántos? ¿Indica algo la cantidad? ¿Se les puede adjudicar alguna función especíVca? ¿Hay evidencia parcial que sugiera que algún nodo funciona de forma anómala y/o no esperada?
	\item ¿Existe una correspondencia entre lo que se conoce de la red y los nodos distinguidos detectados por la herramienta? ¿Es posible usar el criterio de distinción propuesto como método para descubrir el/los Default Gateway/s de la red? ¿Es preciso?
\end{itemize}


\subsection{Experimento 3}

Descripci\'on del experimiento y de las condiciones del experimento

Incluir por lo menos los siguientes gr\'aficos:

\begin{itemize}
	\item Dada la fuente binaria S, mostrar la cantidad de infomación de cada símbolo comparando con la entropía de la fuente y la entropía máxima
	\item Dados los paquetes ARP, muestre mediante un grafo, la red de mensajes ARP subyacente
	\item Dada la fuente S1, mostrar la cantidad de información de cada símbolo comparando con la entropía de la fuente.
\end{itemize}

Responder las siguientes preguntas (solo dejar las respuestas):

\begin{itemize}
	\item ¿La entropía de la fuente S es máxima? ¿Que sugiere esto acerca de la red? ¿Está relacionado con el overhead impuesto por la red debido a los protocolos de control (i.e.: ARP)?
	\item ¿Cómo es el tráVco ARP en la red? ¿Se pueden distinguir nodos? ¿Cuántos? ¿Indica algo la cantidad? ¿Se les puede adjudicar alguna función especíVca? ¿Hay evidencia parcial que sugiera que algún nodo funciona de forma anómala y/o no esperada?
	\item ¿Existe una correspondencia entre lo que se conoce de la red y los nodos distinguidos detectados por la herramienta? ¿Es posible usar el criterio de distinción propuesto como método para descubrir el/los Default Gateway/s de la red? ¿Es preciso?
\end{itemize}
