\section{Introducci\'on}

\subsection{Información}

\par La \textbf{información} de un evento aleatorio $e$ con probabilidad $P(e)$ está dada por:

\begin{equation*}
    I(e) := -log_b(P(e))
\end{equation*}

para una cierta base $b$. 
La elección de dicha base determina la unidad de la información; en este trabajo nos limitaremos a base 2, por lo que la unidad que manejaremos son los \textit{bits}. 

\par La \textbf{entropía} de una variable aleatoria $A$ es la esperanza de la información de $A$, y está dada por:

\begin{equation*}
    H(A) := \sum_{a \in A}{P(a) * I(a)} = - \sum_{a \in A}{P(a) * log(P(a))}
\end{equation*}

\par Una \textbf{fuente} emite mensajes con ciertas probabilidades.
Una \textbf{fuente de memoria nula} es una en la cual la probabilidad de cada mensaje no depende de los mensajes previos; viendo cada mensaje como una variable aleatoria, esto equivale a que sean independientes.
Adicionalmente, si la probabilidad de cada mensaje es constante en el tiempo\footnote{A partir de ahora, asumiremos que lo es.}, estas variables además son idénticamente distribuidas.

\par La entropía de una fuente de memoria nula es la entropía de cada mensaje, que equivale a la información esperada de cada mensaje.

\subsection{Capa de enlace y ARP}

\par Una MAC address es un identificador único para una interfaz utilizada en la capa de enlace.

\par Un frame de una red Ethernet puede ser transmitido de forma \textit{unicast}, es decir para sólo un receptor\footnote{Debido a la naturaleza de Ethernet, se transmite a toda la red, pero los otros receptores normalmente descartarán los frames que no les son destinados.}, especificando como destino su MAC address; o de forma \textit{broadcast}, es decir para todos los dispositivos de la red, marcando como desinto la MAC address FF:FF:FF:FF:FF:FF.

\par En la capa de red se emplea un identificador único denominado dirección de IP.
Para relacionar una de estas direcciones con una MAC address, se emplea el protocolo ARP (\textit{Address Resolution Protocol}).
